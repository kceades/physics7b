\documentclass{article}

\usepackage{array}

\begin{document}

\title{Coloumb's Law, Problems}
\author{GSI: Caleb Eades}
\date{9/27}
\maketitle

\section{Coulomb's Law}

\subsection{Charges in a Bowl}

Two identical charges, each with mass $m$, are at rest on the surface of a hemispherical bowl of radius $R$, separated by an angle $\theta$.
\begin{itemize}
	\item[(a)] Find the charge $Q$ on each of the charges.
	\item[(b)] Is this equilibrium stable? If so, calculate the frequency of small oscillations about it.
\end{itemize}

(\textit{Source: modified from physics-prep.com})

\subsection{Oh Charge, Where Art Thou?}

Two point charges are located on the $x$ axis. They are both positive, bu the one located at $x=0$ has a charge of $q$ while the one located at $x=L$ has a charge of $4q$. If a third charge is placed on the $x$ axis in between the two charges so that the net force on ANY of the charges is zero, determine the magnitude of the third charge and its location.

(\textit{Source: physics-prep.com})

\subsection{Return of the Spring}

A spring with spring constant $k_s$ and rest length $L$ has positive charges $Q$ attached to either end.
\begin{itemize}
	\item[(a)] Find an equation that will determine the length $D$ of the spring, once the charges have come to rest.
	\item[(b)] Repeat part (a), this time assuming that the charges on either end are both \textit{negative}.
	\item[(c)] Repeat again, this time assuming that the charges on either end have \textit{opposite} signs.
\end{itemize}

(\textit{Source: workbook })

\subsection{A Balancing Act}

A charge $q$ hangs on the end of a string while another charge $-Q$ of mass $m$ is brought beneath it.
\begin{itemize}
	\item[(a)] At what distance $d$ below the hanging mass is the charge $-Q$ in equilibrium?
	\item[(b)] Is this equilibrium stable? If so, find the frequency of small oscillations about it.
\end{itemize}

\subsection{Dipoles}

(Challenge) Find the electric field due to a dipole located at the origin both along its axis and in the plane perpendicular to its axis. (Hint: find the field from two point charges with charge $q$ located at $y=d/2$ and charge $-q$ located at $y=-d/2$ and then take the limit as $d$ goes to zero.)

\end{document}
