\documentclass{article}

\begin{document}
	
\title{Math Review Problems}
\author{GSI: Caleb Eades}
\maketitle

\section{Dimensional Analysis}

\subsection{Checking the answer, part 1}

If I were solving for the energy fo a pendulum and got the formula $E=lg$, should I be concerned?

\subsection{Checking the answer, part 2}

When solving of the force acting in the y-direction of a block falling down a ramp, I got $F_y=mgysin(\alpha) + \mu mgcos(y\alpha)$. What is everything wrong with this equation?

\subsection{Inferring units}

Suppose we had an equation for a distance $D$ that relates to the acceleration $a$ and velocity $v$ by
\begin{equation}
D = \frac{5v}{3n^2} + \frac{a}{wv}e^{ha},
\end{equation}
where $n$, $w$ and $h$ are all constants. What can we infer about the units of these constants?

\newpage

\section{Taylor Series}

\subsection{Finding Maclaurin Series}

Find the Maclaurin series for $f(x) = cos(x)$ and $f(x)=e^x$. What is the Maclaurin series for $f(x)=(e^x)^2$? What about $f(x)=ln(x)$?

\subsection{Approximating using Taylor Series}

Approximate $1/2.0001^2$.

\subsection{Special relativity}

Special relativity extends classical mechanics so that it can describe objects traveling near the speed of light, $c$, which is $3\times10^8$ m/s. In special relativity, the energy of a free object is $\gamma mc^2$ (this is the famous $E=mc^2$) and the momentum is $\gamma mv$, where $\gamma = \frac{1}{1-v^2/c^2}$. Is this consistent with the relations for energy and momentum you learned about in 7A? For a car traveling at 60 mph, how big is the relativistic correction?

\subsection{A harder approximation}

The acceleration due ot gravity is approximately $g$ near the Earth's surface, where $g$ is approximating $\frac{GM}{R^2}$. How much does this change at the top of Mount Everest, to first order? What order TAylor series approximation is needed to determine the gravitational acceleration at the edge of the thermosphere (where some satellites orbit) to within one percent? (Useful information: radius of the Eart is approximately 6400 km, height of Everest is approximately 8.5 km, and the edge fo the thermosphere is approximately 600 km above Earth. $G=6.674\times10^{-11}$ $m^3 kg^{-1} s^{-2}$, $M=5.972\times10^{24}$ kg.)

\end{document}