\documentclass{article}

\usepackage{amsmath}

\begin{document}
	
\title{Math Review Solutions}
\author{GSI: Caleb Eades}
\maketitle

\section{Dimensional Analysis}

\subsection{Checking the answer, part 1}

Yes you should be concerned, because $lg$ has units of Newtons (units of force), not Joules (units of energy).

\subsection{Checking the answer, part 2}

When solving of the force acting in the y-direction of a block falling down a ramp, I got $F_y=mgysin(\alpha) + \mu mgcos(y\alpha)$. What is everything wrong with this equation?

\begin{itemize}
	\item $mgy$ is units of energy or work (Joules), not of force (Newtons).
	\item $\alpha$ must have units of radians based on the first term, so then in the second term, you cannot take a cosine of $y\alpha$ because that has units of meters.
	\item It is just outright the wrong answer for force acting in the y-direction for a block falling down a ramp, but for the purposes here, we were concerned mostly with the units issues.
\end{itemize}

\subsection{Inferring units}

Suppose we had an equation for a distance $D$ that relates to the acceleration $a$ and velocity $v$ by
\begin{equation}
D = \frac{5v}{3n^2} + \frac{a}{wv}e^{ha},
\end{equation}
where $n$, $w$ and $h$ are all constants. What can we infer about the units of these constants?

From the first term, we can infer that $n$ has units of seconds to the negative one-half power, or $s^{-1/2}$. From the second term, the argument of the exponential must be unitless, so $h$ has units of inverse acceleration or $s^2 m^{-1}$. From the coefficient of the exponential, $w$ has units of $m^{-1}s^{-1}$.

\newpage

\section{Taylor Series}

\subsection{Finding Maclaurin Series}

\begin{equation}
cos(x) = 1 - \frac{x^2}{2!} + \frac{x^4}{4!} - ...
\end{equation}

\begin{equation}
e^x = 1 + x + \frac{x^2}{2!} + \frac{x^3}{3!} + ...
\end{equation}

\begin{align*}
(e^x)^2 &= e^{2x} \\
&= 1 + (2x) + \frac{(2x)^2}{2!} + ... \\
&= 1 + 2x + \frac{2^2x^2}{2!} + \frac{2^3x^3}{3!} + ...
\end{align*}

We cannot find a Maclaurin series for $ln(x)$ because the derivatives are not well defined at $x=0$.

\subsection{Approximating using Taylor Series}

\begin{align*}
1/2.0001^2 &= (2.0001)^{-2} \\
&= (2 + 0.0001)^{-2} \\
&= 2^{-2}(1+0.00005)^{-2} \\
&\approx \frac{1}{4}(1-0.0001) \\
&= \frac{1}{4}(0.9999) \\
&= 0.249975
\end{align*}

\subsection{Special relativity}

For small v,
\begin{equation}
\gamma = (1-v^2/c^2)^{-1/2} \approx 1 + \frac{1}{2}\frac{v^2}{c^2}
\end{equation}
This is a second-order correction for momentum:
\begin{equation}
p \approx \left(1 + \frac{1}{2}\frac{v^2}{c^2}\right) mv \approx mv
\end{equation}
Whereas for energy $mc^2$ is a constant so we must keep the second-order correction. Ignoring the constant since we are primarily concerned with energy differences (we can define the 'zero' anywhere), we get:
\begin{align*}
E &\approx \left(1 + \frac{1}{2}\frac{v^2}{c^2}\right) mc^2 \\
&= mc^2 + \frac{1}{2} mv^2 \\
&\rightarrow \frac{1}{2} mv^2
\end{align*}
For a car traveling at 60 mph,
\begin{align*}
v &= 60 \frac{miles}{1 hr} \times \frac{1 hr}{3600 s} \times \frac{5280 ft}{1 mile} \times \frac{12 in}{1 ft} \times \frac{2.54 cm}{1 in} \times \frac{1 m}{100 cm} \\
&\approx 26.82 \frac{m}{s}
\end{align*}
Then,
\begin{align*}
(p_{rel}-p_{cla})/p_{cla} &= \gamma - 1 \\
&= \sqrt{\frac{1}{1-(26.82 m/s)^2/(3\times10^8 m/s)^2}} - 1 \\
&\approx 4 \times 10^{-15}
\end{align*}
or more simply using the binomial approximation,
\begin{equation}
\gamma - 1 \approx \frac{1}{2}\frac{v^2}{c^2} \approx 4 \times 10^{-15}
\end{equation}
Bonus thought: note that the binomial approximation is really accurate here. This is for two reasons, one being $v\ll c$ and the other being it is a second-order approximation in $\left(\frac{v}{c}\right)$, making it more precise than most first-order binomial approximations.

\subsection{A harder approximation}

\begin{align*}
GM(R+h)^{-2} &= \frac{GM}{R^2}(1+h/R)^{-2} \\
&= g(1 + (-2)(h/R) + (-2)(-3)(h/R)^2/(2!) + ...) \\
&\approx g(1-2h/R)
\end{align*}
On top of Everest,
\begin{equation}
\frac{-2gh/R}{g} = -2(8.5/6400) \approx -0.003
\end{equation}
At the edge of the thermosphere,
\begin{align*}
\frac{GM}{(R+h)^2} &= \frac{(6.674\times10^{-11} m^3 kg^{-1} s^{-2})(5.972\times 10^{24} kg)}{(7000 \times 10^3 m)^2} \\
&\approx 8.134 m s^{-2}
\end{align*}
A cone-percent interval means we are looking for something in the window $(0.99a,1.01a)=(8.053,8.215) m s^{-2}$. The Taylor Series approximations in successive orders are:
\begin{align*}
P_0 &= g \approx 9.731 m s^{-2} \\
P_1 &= g(1-2h/R) \approx 7.906 m s^{-2} \\
P_2 &= g(1-2h/R+3(h/R)^2) \approx 8.162 m s^{-2}
\end{align*}
Hence we only need a second-order Taylor series approximation to get within one percent (and it is significantly better than one percent at that).

\end{document}