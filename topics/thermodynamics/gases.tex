\documentclass{article}

\begin{document}
	
\title{Probability Distributions and Gases}
\author{GSI: Caleb Eades}
\maketitle

\section{Probability Distributions}

\subsection{Statistics on distributions}

Consider the following probability distribution for velocities of atoms in a gas:
\begin{equation}
f(v) = C(a-bv^2)
\end{equation}
for $-\sqrt{a/b} < v < \sqrt{a/b}$ and
\begin{equation}
f(v) = 0
\end{equation}
otherwise, where $a$ and $b$ are fixed positive constants and $C$ is an unknown constant.
\begin{itemize}
	\item[(a)] Find the value fo $C$ which ensures that all atoms in the gas have some velocity.
	\item[(b)] Find the average velocity $v_{avg}$ of atoms in this gas.
	\item[(c)] Find the average value of $v^2$ of atoms in this gas.
	\item[(d)] What is the number distribution of this gas (a distribution of hte number of atoms with some velocity)?
\end{itemize}

\subsection{Concrete numbers}

You have 10 gas molecules in a box. At one moment, two have a speed of 10 m/s, four have a speed of 12 m/s, two have a speed of 14 m/s, one has a speed of 15 m/s and one has a speed of 17 m/s. The gas molecules each have a mass $m$.
\begin{itemize}
	\item[(a)] Calculate the average speed and the rms speed.
	\item[(b)] Using its strict definition, what would the ``temperature'' be for this theoretical distribution? Leave your answer in terms of $m$ and $k_B$.
\end{itemize}

\subsection{Setting up expressions}

At extremely low temperatures, the usual methods of cooling a gas
do not work. Instead, a form of evaporative cooling is used, and it is
based on the fact that for a Maxwell speed distribution the $p_N = 10\%$
of the gas molecules with the highest speed carry $p_E = 28\%$ of the total
energy of the gas. By removing these atoms and waiting for the gas
to come back gas to come back to equilibrium, you can cool the gas a
fixed amount in each evaporation cycle.
\begin{itemize}
	\item[(a)] The fastest $p_N = 10\%$ of molecules will correspond to molecules
with speeds larger than some value α. Write down (but do not
solve) the equation that you would use to determine $\alpha$.

This is simple to write down, but hard to solve so do not try to
solve it!
	\item[(b)] Write down (but do not evaluate) the expression to determine
what percentage of the energy the molecules with speeds larger
than $\alpha$ carry. That is, relate the $p_E = 28\%$ above to the Maxwell
distribution.
	\item[(c)] Suppose a gas has initial temperature $T_0$. What is the temperature $T_1$ of the gas afer one cycle?
	\item[(d)] Suppose you wanted to cool the gas to below $T_0/2$. What is the minimum number of cycles it will take?
\end{itemize}

\subsection{Puddles!}

Why do puddles evaporate, even if the temperature is much colder than the boiling point of water? Why do sealed jars never evaporate?

\subsection{Building intuition}

Plot a typical Maxwell Distribution for some value of $N$ and $T$. What would it look like if you increased the termpature, keeping $N$ constant? What would it look like if you increased the number of molecules, but kept $T$ constant?

\end{document}