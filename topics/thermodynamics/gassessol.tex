\documentclass{article}

\usepackage{amsmath}

\begin{document}
	
\title{Probability Distributions and Gasses Solutions}
\author{GSI: Caleb Eades}
\maketitle

\section{Ideal Gas Law}

\subsection{Pressured balloons}

The lake adds a pressure $\rho gh$ where $h$ is the height beneath the surface. So letting $D$ bet the depth of the lake,
\begin{align*}
PV &= nRT \\
P' &= nRT'/V' \\
P_{balloon} + P_{depth} &= nRT'/V' \\
nRT/V + \rho gh &= nRT'/V' \\
\rho gh &= nR(T'/V'-T/V) \\
h &= \frac{nR}{\rho g}(T'/V'-T/V) \\
&= \frac{PV}{\rho gT}(T'/V'-T/V) \\
&= \frac{P}{\rho g}\left(\frac{VT'}{TV'}-1\right)
\end{align*}

\subsection{Pressure gauges}

Have fun!

\subsection{Estimations with the ideal gas law}

In a standard room, $P\approx 1 atm$, $T \approx 300 K$, $R\approx 8 J mol^{-1} K^{-1}$, $V\approx (5 m)^2(4 m) = 100 m^3$, and $1 atm\approx 10^5 N m^{-2}$. So putting all this together,
\begin{equation}
n = \frac{PV}{RT} \approx \frac{(10^5 N m^{-2})(100 m^3)}{(8 N m mol^{-1} K^{-1})(300 K)} \approx \frac{100}{25}\times 10^3 mol = 4000 moles
\end{equation}
Multiplying by Avogadro's number $\left(6.022\times10^{23}\right)$, we get
\begin{equation}
N\approx 2.5\times10^{26}
\end{equation}

\subsection{Partitioned boxes}

\begin{itemize}
	\item[(a)] The gas expands into the remainder of the box that used to be a vacuum. So $V/2\rightarrow V$ and by the Ideal Gas Law,
	\begin{align*}
	P_0\frac{V}{2}&=nRT_0 \\
	P_fV &= nRT_f
	\end{align*}
	The termpreature remains unaltered, however, because nothing has been doen to increase or decrease the kinetic energy of the molecules, so $T_f=T_0$ and therefore by the two equations above, $P_f=P_0/2$.
	\item[(b)] This is trickier because along the way, something has to hold the wall in place, which does negative work on the gas (you can also think about the gas doing work on the wall to expand and push it outwards), so the temperature no longer remains constant. Hence, $T$ decreases, $V$ increases ($V/2\rightarrow V$ as before) and $P$ decreases (by more than half now).
\end{itemize}

\end{document}