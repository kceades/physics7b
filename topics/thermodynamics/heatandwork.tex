\documentclass{article}

\begin{document}
	
\title{Heat and Work}
\author{GSI: Caleb Eades}
\date{9/11}
\maketitle

\section{Derivations/Proofs}

\subsection{Relationship of Molar Specific Heats}

Use the first law of thermodynamics to show that $C_P = C_V + R$. (\textit{Source: Supplement})

\subsection{Equipartition Theorem and Molar Specific Heats}

Use the equipartition theorem and the first law of thermodynamics for a constant volume process to show that $C_V = \frac{d}{2}R$. (\textit{Source: Supplement})

\newpage

\section{General Problems from Chapter 19}

\subsection{Bullets and Blocks}

A $2.0$ g lead bullet moving at a speed of $200$ m/s becomes embedded in a $2.0$ kg wooden block of a ballistic pendulum. Calculate the rise in temperature of the bullet, assuming that all the heat generated raises the bullet's temperature. (\textit{Source: Halliday and Resnick})

\subsection{Thinking Through Every Effect}

Consider two identical homogeneous balls with the same initial temperature. One of them is at rest on a horizontal plane, while the second hangs from a thread. The same quantities of heat are supplied to both balls. Are the final temperatures of the balls the same or not? If not, calculate the temperature difference. (\textit{Source: International Physics Olympiad})

\subsection{Fun with Van der Waals (isothermal)}

The Van der Waals equation of state is an alternative, more realistic, model for gases than the ideal gas law (you just need the formula for this problem). It has two extra constants $a$ and $B$ such that
\begin{equation}
(P+\frac{a}{(V/n)^2})(\frac{V}{n}-b) = RT.
\end{equation}
Calculate the work done in an isothermal expansion of a Van der Waals gas from $(P_1, V_1)$ to $P_2, V_2$. (\textit{Source: Supplement})

\subsection{Fun with Pistons (adiabatic)}

Consider a gas in a closed container that is pushed to a smaller space with a piston. An adiabatic process is one in which the work-energy theorem holds for the gas; even though we had to push on the piston to push in the gas to do work on it, all the work went straight to the gas and wasn't lost in the random motion of particles in the piston.
\begin{itemize}
	\item[(a)] Show that in this scenario $PV = (\gamma-1)U$.
	\item[(b)] Differentiate this equation with respect to volume. Do not assume $P$ is a constant, or that you know how $P$ changes with $V$. Therefore you'll need to have $dP/dV$ in your solution.
	\item[(c)] Multiply your solution to (b) by $dV$. Use the first law of thermodynamics and the fact that there is no heat added in adiabatic processes to solve for $P$ as a function of $V$. This is the adiabatic formula that you learned in class.
\end{itemize}
(\textit{Source: Supplement})

\subsection{Fun with Cycles (isobaric/isovolumetric)}

Consider the following two-step process. Heat is allowed to flow out of an ideal gas at constant volume so that its pressure drops from $2.2$ atm to $1.4$ atm. Then the gas expands at constant pressure, from a volume of $5.9$ L to $9.3$ L, where the temperature reaches its original value. Calculate (a) the total work done by the gas in the process, and (b) the change in internal energy of the gas in the process, and (c) the total heat flow into or out of the gas. (\textit{Source: Giancoli 19.31})

\end{document}