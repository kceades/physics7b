\documentclass{article}

\begin{document}
	
\title{Ideal Gas Problems}
\author{GSI: Caleb Eades}
\maketitle

\section{Ideal Gas Law}

\subsection{Pressured balloons}

A balloon is filled at ta pressure and temperature $P$ and $T$, respectively, to a volume of $V$. Then, it is taken to the bottom of a lake, where the temperature is $T'$. The balloon's volume is measured to be $V'$. How deep is the lake? Suppose that the density of water is a constant value $\rho$.

\subsection{Pressure gauges}

How could you measure the pressure of a gas?

\subsection{Estimations with the ideal gas law}

Estimate the number of air molecules in an average-sized room.

\subsection{Partitioned boxes}

Suppose we have a box with total volume $V$ and we have a partition in the middle with a monoatomic ideal gas at pressure $P_0$ and temperature $T_0$ on the left with a vacuum on the right.
\begin{itemize}
	\item[(a)] If we suddenly remove the partition, what happens? What is the final temperature and pressure of the gas?
	\item[(b)] (Challenge) If we instead slowly move the partition to the right until it joins the rightmost wall, what happens to the gas along the way? (Qualitatively) What is the final temperature and pressure of the gas?
\end{itemize}

\end{document}