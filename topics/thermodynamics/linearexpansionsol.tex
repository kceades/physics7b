\documentclass{article}

\usepackage{amsmath}

\begin{document}
	
\title{Linear Expansion Solutions}
\author{GSI: Caleb Eades}
\date{8/28}
\maketitle

\section{Thermal Expansions}

\subsection{Three Dimensions}

The initial volume:
\begin{equation}
V_0 = L_0 H_0 W_0
\end{equation}
The volume after isotropic expansion:
\begin{align*}
V &= L_0(1+\alpha\Delta T)H_0(1 + \alpha\Delta T)W_0(1 + \alpha\Delta T) \\
&= L_0H_0W_0(1+\alpha\Delta T)^3 \\
&= V_0(1+3\alpha\Delta T + 3\alpha^2\Delta T^2 + \alpha^3\Delta T^3) \\
&\approx V_0(1+3\alpha\Delta T)
\end{align*}
for small $\Delta T$. So then $\beta = 3\alpha$ is the volume expansion coefficient.

Similarly, for anisotropic expansion,
\begin{align*}
V &= V_0(1+\alpha_L\Delta T)(1 + \alpha_{HW}\Delta T)^2 \\
&= V_0(1+\alpha_L\Delta T)(1+2\alpha_{HW}\Delta T + \alpha^2_{HW}\Delta T^2) \\
&= V_0(1+(\alpha_L + 2\alpha_{HW}\Delta T + ...) \\
&\approx V_0(1+(\alpha_L + 2\alpha_{HW})\Delta T)
\end{align*}
for small $\Delta T$. So the volume expansion coefficient is $\beta = \alpha_L + 2\alpha_{HW}$.

\subsection{Expanding/contracting holes}

The hole gets larger! There are two ways to view this.

View 1:

Concentric thin circles are like lines. So a circle of radius $r$ and circumference $C=2\pi r$ will expand to $C=2\pi r(1+\alpha\Delta T)$. Every small circle gets bigger so the hole of the annulus grows larger!
\begin{align*}
a &= a_0(1+\alpha\Delta T) \\
b - a &= (b_0-a_0)(1+\alpha\Delta T) \\
\pi(b^2-a^2) &\approx \pi(b_0^2-a_0^2)(1+2\alpha\Delta T)
\end{align*}

View 2:

Radial distance is just a length dimension that should expand linearly as $r=r_0(1+\alpha\Delta T)$. The other results follow.

\subsection{Becoming an experimenter}

Have fun!

\end{document}