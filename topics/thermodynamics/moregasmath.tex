\documentclass{article}

\begin{document}
	
\title{More Gas Math, Problems}
\author{GSI: Caleb Eades}
\date{9/4}
\maketitle

\section{More Statistics}

\subsection{Continuous vs Discrete}

Consider a system of $N>>1$ ideal gas particles moving with speeds determined by a distribution, either continuous or discrete. For the continuous distribution, let $f(v) = Asin(\pi v)$ for $v\in(0,1)$ m/s. For the discrete distribution, the particles can have velocity $v=0$ m/s or $v=1$ m/s with probability of $0$, $v=1/4$ m/s or $v=3/4$ m/s with probability $B/\sqrt{2}$, or $v=1/2$ m/s with probability of $B$.

\begin{itemize}
	\item[(a)] Find $A$ and $B$ such that the distributions, individually, are each valid probability distributions. Do $A$ and $B$ have units associated with them? What about $\pi$ in $f(v)$?
	\item[(b)] For each distribution, calculate $<v> = \bar{v}$, $v_{rms}$, $\sigma$, and $v_p$.
	\item[(c)] What is the temperature of the gas?
	\item[(d)] Comment on any discrepancies (or lack thereof) between the statistics for this discrete and continuous distribution. (e.g., Why did (or didn't) these discrepancies arise?)
\end{itemize}

\subsection{Fun with Maxwell}

Show that the total energy of a gas governed by the Maxwell distribution is given by
\begin{equation}
E = \left<\frac{1}{2}mv^2\right>=\frac{3}{2}k_BT.
\end{equation}
(Challenge: do this two different ways.)

\end{document}