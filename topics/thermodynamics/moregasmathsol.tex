\documentclass{article}

\usepackage{amsmath}

\begin{document}
	
\title{More Gas Math, Solutions}
\author{GSI: Caleb Eades}
\date{9/4}
\maketitle

\section{More Statistics}

\subsection{Continuous vs Discrete}

\begin{itemize}
	\item[(a)] First, we solve for $A$:
	\begin{align*}
	\int_0^{\infty} f(v)dv &= \int_0^1 f(v)dv \\
	&= A\int_0^1 sin(\pi v)|_0^1 \\
	&= \frac{-A}{\pi}cos(\pi v)|_0^1 \\
	&= \frac{2A}{\pi}
	\end{align*}
	We mush have $\int_0^{\infty} f(v)dv = 1$, so $A = \frac{\pi}{2}$. Now, solving for $B$:
	\begin{align*}
	P(v=0) + P(v=1/4) + P(v=1/2) + P(v=3/4) + P(v=1) &= 1 \\
	0 + B/\sqrt{2} + B + B/\sqrt{2} + 0 &= 1 \\
	B(1 + \sqrt{2}) &= 1 \\
	B &= \frac{1}{1+\sqrt{2}}
	\end{align*}
	For the quesiton of units, $A$ has units of seconds/meter, $B$ is unitless, and $\pi$ has units of seconds/meter. If this last one confuses you, you could equivalently have a ``ghost'' one in there with units of seconds/meter and let $\pi$ be unitless.
	\item[(b)] For each distribution, we calculate $<v> = \bar{v}$, $v_{rms}$, $\sigma$, and $v_p$. Starting with $A$:
	\begin{align*}
	<v> &= \int_0^{\infty} vf(v)dv \\
	&= \int_0^1 \frac{\pi}{2}vsin(\pi v)dv \\
	&= \frac{\pi}{2}\left[\frac{-v}{\pi}cos(\pi v)|_0^1 + \int_0^1\frac{1}{\pi}cos(\pi v)dv\right] \\
	&= \frac{1}{2}\left[-vcos(\pi v) + \frac{1}{\pi} sin(\pi v)|0^1\right] \\
	&= \frac{1}{2} m/s
	\end{align*}
	\begin{align*}
	<v^2> &= \int_0^{\infty} v^2f(v)dv \\
	&= \int_0^1 \frac{\pi}{2}v^2sin(\pi v)dv \\
	&= \frac{\pi}{2}\left[\frac{-v^2}{\pi}cos(\pi v)|_0^1 + \int_0^1\frac{2}{\pi}vcos(\pi v)dv\right] \\
	&= \frac{-v^2}{2}cos(\pi v)|_0^1 + \frac{v}{\pi} sin(\pi v) + \frac{1}{\pi^2}cos(\pi v)|_0^1 \\
	&= \frac{1}{2} - \frac{2}{\pi^2} \\
	&= \frac{\pi^2-4}{2\pi^2} \\
	&\approx 0.30 m^2 s^{-2}
	\end{align*}
	\begin{align*}
	\sigma &= \sqrt{<v^2>-<v>^2} \\
	&= \sqrt{\frac{1}{2}-\frac{2}{\pi^2}-\frac{1}{4}} \\
	&= \sqrt{\frac{1}{4}-\frac{2}{\pi^2}} \\
	&= \frac{1}{2\pi}\sqrt{\pi^2-8} \\
	&\approx 0.22 m/s
	\end{align*}
	Lastly, taking derivatives and finding the local maximum between 0 and 1:
	\begin{equation}
	f'(v) = \frac{\pi^2}{2}cos(\pi v) \implies v_p=\frac{1}{2} m/s
	\end{equation}
	where this can be verified with a second derivative test if you want.
	Now, for $B$, we do the same thing in the discrete case:
	\begin{align*}
	<v> = \frac{1}{1+\sqrt{2}}\left(\frac{1}{\sqrt{2}}\frac{1}{4} + \frac{1}{\sqrt{2}}\frac{3}{4} + \frac{1}{2}\right) \\
	&= \frac{1}{1+\sqrt{2}}\left(\frac{1}{\sqrt{2}}+\frac{1}{2}\right) \\
	&= \frac{1}{1+\sqrt{2}}\frac{1+\sqrt{2}}{2} \\
	&= \frac{1}{2} m/s
	\end{align*}
	\begin{align*}
	<v^2> = \frac{1}{1+\sqrt{2}}\left(\frac{1}{\sqrt{2}}\frac{1}{16} + \frac{1}{\sqrt{2}}\frac{9}{16} + \frac{1}{4}\right) \\
	&= \frac{1}{1+\sqrt{2}}\frac{5+2\sqrt{2}}{8\sqrt{2}} \\
	&\approx 0.29 m^2 s^{-2} \\
	\end{align*}
	\begin{align*}
	\sigma &= \sqrt{<v^2>-<v>^2} \\
	\approx 0.19 m/s
	\end{align*}
	Lastly, $v_p = \frac{1}{2}$ m/s since it is simply the speed with the highest probability.
	\item[(c)] A:
	\begin{align*}
	<v^2> &= 3\frac{k}{m}T \\
	T &= \frac{m}{3k}<v^2> \\
	&\approx \frac{1}{10}{m}{k} (m^2 s^{-2})
	\end{align*}
	B:
	\begin{equation}
	T = \frac{m}{3k}<v^2> \approx 0.096\frac{m}{k} (m^2 s^{-2})
	\end{equation}
	\item[(d)] Even though the discrete distribution was taken from samples of the continuous one, the statistics differe slightly. Nonetheless, for only five samples, the difference are fairly small: $\sigma_A\approx0.22$ m/s whereas $\sigma_B\approx0.19$ m/s. Hence, B has a tighter distribution, which makes sense since the tails of A are ignored with the resampling.
	
	In general, you have to be careful with how you resample a continuous distribution to get a discrete spectrum that is more computable (in ``real-life'' cases where the continuous distribution does not follow a pretty function or is unknown entirely).
\end{itemize}

\subsection{Fun with Maxwell}

\begin{itemize}
	\item[(1)] Method 1 (following pgs. 477-78 of Giancoli): With wall collisions, $\Delta (mv) = 2mv_x$ on the x-direction walls. In a box of dimensions $L$, the time between collisions in this direction is $\Delta t = 2L/v_x$, so
	\begin{equation}
	F = \frac{\Delta (mv)}{\Delta t} = \frac{mv_x^2}{L}
	\end{equation}
	for the wall's force exerted on a molecule in a collision (averaged). Summing over all the molecules,
	\begin{equation}
	F_{total} = \frac{m}{L}\left(v_{x1}^2 + \cdot + v_{xN}^2\right)
	\end{equation}
	Mulitplying by $N/N$ and noting that $\bar{v_x^2} = \frac{v_{x1}^2+ \cdot +v_{xN}^2}{N}$, we have
	\begin{equation}
	F = \frac{m}{L}N\bar{v_x^2}
	\end{equation}
	Now, $\bar{v^2}=3\bar{v_x^2}$, so with $P = \frac{F}{A} = \frac{F}{L^2}$, we have
	\begin{equation}
	P = \frac{1}{3}\frac{Nm\bar{v^2}}{L^3} = \frac{1}{3}\frac{Nm\bar{v^2}}{V}
	\end{equation}
	Rearranging, $PV = \frac{2}{3}N\left(\frac{1}{2}m\bar{v^2}\right)$. From the Ideal Gas Law, $PV = NkT$, so
	\begin{equation}
	\frac{2}{3}E = kT \implies E = <\frac{1}{2}m\bar{v^2}> = \frac{3}{2}kT
	\end{equation}
	\item[(2)] Method 2 (from the Maxwell distribution): Using $f(v)$ from the Maxwell distribution and $<v^2> = \int_0^{\infty} v^2f(v)dv$, we have
	\begin{equation}
	<v^2> = \int_0^{\infty} Av^4\exp{-Bv^2} dv
	\end{equation}
	where $A = 4\pi\left(\frac{m}{2\pi kT}\right)^{3/2}$ and $B = \frac{1}{2}\frac{m}{kT}$. Observe that
	\begin{equation}
	v^4\exp{-Bv^2} = \frac{d^2}{dB^2}\left(\exp{-Bv^2}\right)
	\end{equation}
	So we can use this to rewrite (and using the formula for the integral of a Gaussian on the second step)
	\begin{align*}
	<v^2> &= A\{frac{d^2}{dB^2}\int_0^{\infty} \exp{-Bv^2}dv \\
	&= A\{frac{d^2}{dB^2}\left(\frac{1}{2}\sqrt{\frac{\pi}{B}}\right) \\
	&= \frac{1}{2}\sqrt{\pi}\frac{-1}{2}\frac{-3}{2}B^{-5/2} \\
	&= \frac{3\sqrt{\pi}}{8}AB^{-5/2}
	\end{align*}
	Plugging in the constants again,
	\begin{align*}
	<v^2> &= \frac{3\sqrt{\pi}}{8}\times 4\pi\left(\frac{m}{2\pi kT}\right)^{3/2}\times \left(\frac{1}{2}\frac{m}{kT}\right)^{-5/2} \\
	&= 3\left(\frac{m}{kT}\right)^{3/2}\left(\frac{kT}{m}\right)^{5/2} \\
	&= 3\frac{kT}{m}
	\end{align*}
	Hence,
	\begin{equation}
	<\frac{1}{2}mv^2 = \frac{3}{2}kT
	\end{equation}
\end{itemize}

\end{document}