\documentclass{article}

\begin{document}
	
\title{Non-Ideal Gases, Problems}
\author{GSI: Caleb Eades}
\date{9/4}
\maketitle

\section{Gases: Ideal, Real and van der Waals}

\subsection{Conceptual Questions}

\begin{itemize}
	\item[(a)] If a container of gas is at rest, the average velocity of molecules must be zero. Yet the average speed is not zero. Explain. (\textit{Source: Giancoli Problem 18.11})
	\item[(b)] Alcohol evaporates more quickly than water at room temperature. What can you infer about the molecular properties of one relative to the other? (\textit{Source: Giancoli Problem 18.15})
\end{itemize}

\subsection{Escape Velocities}

The escape speed from the Earth is $1.12\times10^4$ m/s, so that a gas molecule traveling away from Earth near the outer boundary of the Earth's atmosphere would, at this speed, be able to escape from the Earth's gravitational field and be lost to the atmosphere. At what temperature is the average speed of (a) oxygen molecules, and (b) helium atoms equal to $1.12\times10^4$ m/s? (c) Can you explain why our atmosphere contains oxygen but not helium? (\textit{Source: Giancoli Problem 18.60})

\subsection{Fermi-ish Validation of Ideal Gas Law}

Assuming a typical nitrogen or oxygen molecule is about $0.3$ nm in diameter, what percent of the room you are sitting in is taken up by the volume of the molecules themselves? (\textit{Source: Giancoli Problem 18.65})

\subsection{Reformulation of Pressure}

Show that the rms speed of molecules in a gas is given by $v_{\mathrm{rms}} = \sqrt{3P/\rho}$, where $P$ is the pressure of the gas and $\rho$ is the gas density. (\textit{Source: Prof. Adrian Lee, Fall 2009 Phys. 7B 1st Midterm})

\subsection{Pressure from Other Things}

During a hailstorm, hailstones with an average mass of 2g and a speed of 15 m/s strike a window pane at a $45^{o}$ angle. The area of the window is $0.5 \mathrm{m}^2$ and the hailstones hit it at a rate of 30 per second. What average pressure do they exert on the window? How does this compare to the pressure of the atmosphere? (\textit{Source: Schroeder - Thermal Physics Problem 1.21})

\end{document}