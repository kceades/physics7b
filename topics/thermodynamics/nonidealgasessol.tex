\documentclass{article}

\usepackage{amsmath}

\begin{document}
	
\title{Non-Ideal Gases, Solutions}
\author{GSI: Caleb Eades}
\date{9/4}
\maketitle

\section{Gases: Ideal, Real and van der Waals}

\subsection{Conceptual Questions}

\begin{itemize}
	\item[(a)] Average velocity is zero but speed looks at the magnitude (some move left and some move right, for example). So overall while average speed is nonzero, average velocity will be.
	\item[(b)] We can infer that there is less inter-molecular bonding since there is a lower ``escape velocity'' for the alcohol.
\end{itemize}

\subsection{Escape Velocities}

First off, in a Maxwell Distribution,
\begin{equation}
\bar{v}=\sqrt{\frac{8}{\pi}\frac{kT}{m}} \approx 1.60\sqrt{\frac{kT}{m}}
\end{equation}
\begin{itemize}
	\item[(a)] Solving for $T$,
	\begin{align*}
	T &= \frac{m\pi}{8k}\bar{v}^2 \\
	&= \frac{(2*16*1.66\times10^{-27} kg)\pi}{8(1.38\times10^{-23} J/K)}(1.12\times 10^4 m/s)^2 \\
	&\approx 1.9\times10^5 K
	\end{align*}
	\item[(b)] $m_{He} = \frac{1}{8}m_{O_2}$, so
	\begin{equation}
	T_{He} = \frac{1}{8}T_{O_2}\approx2.4\times10^4 K
	\end{equation}
	\item[(c)] At the same temperature, $He$ will go much faster than $O_2$, so more of it will escape, leaving the $O_2$ concentration much higher than $He$.
\end{itemize}

\subsection{Fermi-ish Validation of Ideal Gas Law}

Starting with the Ideal Gas Law,
\begin{align*}
PV &= nRT \\
n &= \frac{PV}{RT} \\
&\approx \frac{(1\times10^5 Pa)(100 m^3)}{(10 J mol^{-1} K^{-1})(300 K)} \\
&\approx \frac{1}{3}\times10^4 mol \\
&\approx 3\times10^3 mol
\end{align*}
Now, looking at the volume fo the gas,
\begin{align*}
V_{gas} &= \frac{4}{3}\pi (3\times10^{-10} m)^3(3\times 10^3 mol)(6.022\times10^{23}) \\
&\approx 1\times10^{-4} m^3
\end{align*}
Hence, the ratio of the vlume of a gas particle to the volume of the room is
\begin{equation}
\frac{V_{gas}}{V_{room}} \approx \frac{10^{-4}}{10^3} = 10^{-7}
\end{equation}
As we can see, the gas molecules take up less that one part per billion in volume. So the Ideal Gas Law is a decent approximation.

\subsection{Reformulation of Pressure}

From the Ideal Gas Law and the average kinetic energy of a molecule, $PV = NkT$ and $\frac{1}{2} mv^2 = \frac{3}{2}kT$, respectively. Identifying $<v^2> = v_{rms}^2$,
\begin{equation}
v_{rms}^2 = \frac{3kT}{m} = 3\frac{PV/N}{m}
\end{equation}
At this point, it is usefull to recall that $m$ is the mass of a single molecule, so $mN=M$ is the total mass of the gas and hence $\frac{mN}{V} = \frac{M}{V} = \rho$, the density. Putting this together,
\begin{equation}
v_{rms}^2 = 3\frac{P}{\rho} \implies v_{rms} = \sqrt{3P/\rho}
\end{equation}

\subsection{Pressure from Other Things}

Assuming they collide elastically with the window, the total change in momentum is $2mv_x$, with $v_x=\sqrt{2}v/2$. Then,
\begin{align*}
P &= F/A \\
&= \frac{\Delta p}{\Delta t}/A \\
&= \frac{2\sqrt{2}vm}{2}\times\left(30 \frac{1}{second}\right)/A
\end{align*}
where we have identified $\frac{1}{\Delta t}$ as the number of molecules that hit the window per second. So
\begin{align*}
P &= \frac{2*15*2\times10^{-3}}{\sqrt{2}}\times30/(0.5) \\
&\approx 2.6 N m^{-2}
\end{align*}
This is about five orders of magnitude weaker than atmospheric pressure of $1$ atm $\approx1\times10^5 Nm^{-2}$.

\end{document}