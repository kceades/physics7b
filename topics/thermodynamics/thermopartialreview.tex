\documentclass{article}

\usepackage{graphicx}

\begin{document}
	
\title{Thermodynamics: Partial Review, Problems}
\author{GSI: Caleb Eades}
\date{9/6}
\maketitle

\section{Review of Everything Thus Far}

\subsection{2D Gas Laws}

In this problem we will determine the relationship between kinetic energy and temperature for a gas that lives in a 2D world. Consider molecules with mass $m$ that bounce elastically with velocity $\vec{v}$ inside a square of area $L^2$ at temperature $T$.

\begin{itemize}
	\item[(a)] Consider one wall of the container. What is the average time between a molecule's collisions with that wall?
	\item[(b)] What is the average change in momentum of a molecule per collision, $\Delta\vec{p}$ on that wall?
	\item[(c)] What is the average pressure $P$ (force per unit length of the wall) on that wall?
	\item[(d)] If the ideal gas law in two dimensions is $PA = NkT$, what is the relationship between kinetic energy and temperature?
\end{itemize}

\subsection{Thermometers}

I make primitive thermometers consisting of a thin transparent hollow cylinder, that is sealed at one end, and half filled with a liquid. I have experimented with many different liquids. Explain to me why two of my thermometers were hopeless failures:
\begin{itemize}
	\item[(1)] In the first case, I used water.
	\item[(2)] In the second, the column of liquid didn't perceptibly move as the temperature changed, even over a fairly wide range of temperature.
\end{itemize}
(\textit{Source: Summer 2012 First Midterm, Bloxham Problem 1})

\subsection{Expansions}

An alcohol thermometer is made of a cylindrical tube of inner diameter $d_0$ and a bulb of volume $V_0$ at room temperature $T_0$. The volumetric coefficients of thermal expansion are $\beta_{al}$ and $\beta_g$ for the alcohol and glass respectively, with $\beta_{al}>>\beta_g$. Assume that the thickness of the Pyrex glass is negligible. Assume the initial height in the tube is $h$.
\begin{itemize}
	\item[(a)] If the volume of the bulb is much bigger than that of the tube, determine the change in volume of the inside of the thermometer when the temperature is increased from $T_0$ to $T$.
	\item[(b)] Determine the change in volume of the alcohol between the same temperatures (assume the volume of alcohol in the tube is negligible).
	\item[(c)] What is the change in height of the column of alcohol between $T_0$ and $T$?
	\item[(d)] What is the change in height of the column of alcohol between the same temperatures if the change in volume of the alcohol in the tube cannot be neglected?
\end{itemize}
(\textit{Source: Spring 2015 First Midterm, Bordel Problem 1})

\subsection{Mean Free Path}

Re-derive the mean free path for a molecule in an ideal gas.

\subsection{Comparing Gases}

Two identical containers contain two different diatomic gases. While the total mass of the gas in each container is the same, the total number of molecules in $A$ is $N_A$, and the total number of molecules in $B$ is $N_B$. The two gases are at the same temperature, $T$.
\begin{itemize}
	\item[(a)] What is the ratio of RMS velocities, $v_{RMS}^A/v_{RMS}^B$ of the gas in $A$ and $B$?
	\item[(b)] If $P_A$ is the pressure in $A$, and $P_B$ is the pressure in $B$, what is $P_A/P_B$?
	\item[(c)] If we want $v_{RMS}^A = v_{RMS}^B$, by what fraction, $(T_A-T)/T$, should the new temperature, $T_A$, in container $A$ be changed from $T$?
\end{itemize}
(\textit{Source: Fall 2012 First Midterm, Speliotopoulos Problem 3})

\subsection{Pistons and Oscillations}

A cylinder $H$= tall is filled with $n$ mols of an ideal gas at standard temperature and pressure ($T=0^0$C, $P=1$ atm). The top is closed with a tight fitting, frictionless piston of mass $m$ and radius $R$ and the piston is allowed to drop until it is in equilibrium.
\begin{itemize}
	\item[(A)] Find the height of the piston (at equilibrium) assuming that the temperature of the gas does not change as it is compressed.
	\item[(B)] Suppose that the piston is pushed slightly below its equilibrium position and then released. Assuming that the temperature of the gas remains constant, find the frequency of small oscillations of the piston.
\end{itemize}
(\textit{Source: modified from Spring 1999 First Midterm, Packard Problem 2})

\end{document}