\documentclass{article}

\usepackage{array}

\begin{document}

\title{Thermodynamic Processes Problems}
\author{GSI: Caleb Eades}
\date{9/13}
\maketitle

\section{Cycle Through the Cycles}

This problem will develop a useful reference: a list of all quantities associated with thermodynamic processes of ideal gases. Suppose that there are $N$ molecules of an ideal gas with $d$ degrees of freedom (use $\gamma = \frac{d+2}{d}$ where it is more convenient). Suppose the gas starts at $(P_0,V_0)$. Then $T_0 = P_0V_0/(Nk)$. Complete the following table and \textit{draw each process on a P-V diagram}.

\def\arraystretch{2.5}
\begin{table}[h]
	\begin{center}
	\caption{This table is also available in the workbook on pg. 153.}
	
	\begin{tabular}{| >{\Large}c|c|c|c|c|}
		\hline
		\normalsize{\textbf{Quantity}} & \textbf{Isochoric} & \textbf{Isovolumetric} & \textbf{Isothermal} & \textbf{Adiabatic} \\ \hline
		$P_f$             &                   & $P_f$                  &                     &                    \\ \hline
		$V_f$             & $V_f$             &                        & $V_f$               &                    \\ \hline
		$T_f$             &                   &                        &                     & $T_f$              \\ \hline
		$\Delta E$        &                   &                        &                     &                    \\ \hline
		$Q$               &                   &                        &                     &                    \\ \hline
		$W$               &                   &                        &                     &                    \\ \hline
		$\Delta S$        &                   &                        &                     &                    \\ \hline
	\end{tabular}
	\end{center}
\end{table}

\textit{Source: Physics 7B Workbook pg. 153}

\section{Problems}

\subsection{Heat from the Ocean}

It has been proposed to use the thermal gradient of the ocean to drive a heat engine. Suppose that at a certain location the water temperature is $22^o$C at the ocean surface and $4^o$C at the ocean floor.
\begin{itemize}
	\item(a) What is the maximum possible efficiency of an engine operating between these two temperatures?
	\item(b) If the engine is to produce 1 GW of electrical power, what minimum volume of water must be processed (to suck out the heat) in every second?
\end{itemize}

\textit{Source: Schroeder - Thermal Physics problem 4.4}

\subsection{Challenge: Adiabatic Atmosphere}

In an adiabatic atmosphere, $P\rho^{-\gamma}$ is a constant. Show that temperature falls off at a constant rate with height above the earth, and find the rate of this decrease.

\textit{Source: some Feynman physics book problem}

\end{document}
