\documentclass{article}

\begin{document}
	
\title{Week 2, Meeting 1 Problems}
\author{GSI: Caleb Eades}
\date{8/28}
\maketitle

\section{Dimensional Analysis}

\subsection{Checking the answer, part 1}

If I were solving for the energy fo a pendulum and got the formula $E=lg$, should I be concerned?

\subsection{Checking the answer, part 2}

When solving of the force acting in the y-direction of a block falling down a ramp, I got $F_y=mgysin(\alpha) + \mu mgcos(y\alpha)$. What is everything wrong with this equation?

\subsection{Inferring units}

Suppose we had an equation for a distance $D$ that relates to the acceleration $a$ and velocity $v$ by
\begin{equation}
D = \frac{5v}{3n^2} + \frac{a}{wv}e^{ha},
\end{equation}
where $n$, $w$ and $h$ are all constants. What can we infer about the units of these constants?

\newpage

\section{Taylor Series}

\subsection{Finding Maclaurin Series}

Find the Maclaurin series for $f(x) = cos(x)$ and $f(x)=e^x$. What is the Maclaurin series for $f(x)=(e^x)^2$? What about $f(x)=ln(x)$?

\subsection{Approximating using Taylor Series}

Approximate $1/2.0001^2$.

\subsection{Special relativity}

Special relativity extends classical mechanics so that it can describe objects traveling near the speed of light, $c$, which is $3\times10^8$ m/s. In special relativity, the energy of a free object is $\gamma mc^2$ (this is the famous $E=mc^2$) and the momentum is $\gamma mv$, where $\gamma = \sqrt{\frac{1}{1-v^2/c^2}}$. Is this consistent with the relations for energy and momentum you learned about in 7A? For a car traveling at 60 mph, how big is the relativistic correction?

\subsection{A harder approximation}

The acceleration due ot gravity is approximately $g$ near the Earth's surface, where $g$ is approximating $\frac{GM}{R^2}$. How much does this change at the top of Mount Everest, to first order? What order TAylor series approximation is needed to determine the gravitational acceleration at the edge of the thermosphere (where some satellites orbit) to within one percent? (Useful information: radius of the Eart is approximately 6400 km, height of Everest is approximately 8.5 km, and the edge fo the thermosphere is approximately 600 km above Earth. $G=6.674\times10^{-11}$ $m^3 kg^{-1} s^{-2}$, $M=5.972\times10^{24}$ kg.)

\newpage

\section{Thermal Expansions}

\subsection{Three Dimensions}

You have a rectangular prism that has a length $L_0$, a height $H_0$, and a width $W_0$. First, consider that the thermal expansion of the prism is isotropic (equal in all directions). It has a linear expansion coefficient $\alpha$. What is the volume expansion coefficient of the prism for a small temperature change?

Now consider that the thermal expansion of this prism is anisotropic. Its length expands with linear expansion coefficient $\alpha_L$, and its other dimensions expand with linear expansion coefficient $\alpha_{HW}$. What is the volume expansion coefficient of the prism for a small temperature change?

\subsection{Expanding/contracting holes}

If you heat an annulus of inner radius $a_0$ and outer $b_0$, does the hole get larger or smaller? Why? Can you quantify this change? How? Expalin and justify this (hint: there is more that one way/dimension to think this through in).

\subsection{Becoming an experimenter}

How might you measure the coefficient of either linear or volumetric expansion?

\newpage

\section{Ideal Gas Law}

\subsection{Pressured balloons}

A balloon is filled at ta pressure and temperature $P$ and $T$, respectively, to a volume of $V$. Then, it is taken to the bottom of a lake, where the temperature is $T'$. The balloon's volume is measured to be $V'$. How deep is the lake? Suppose that the density of water is a constant value $\rho$.

\subsection{Pressure gauges}

How could you measure the pressure of a gas?

\subsection{Estimations with the ideal gas law}

Estimate the number of air molecules in an average-sized room.

\subsection{Partitioned boxes}

Suppose we have a box with total volume $V$ and we have a partition in the middle with a monoatomic ideal gas at pressure $P_0$ and temperature $T_0$ on the left with a vacuum on the right.
\begin{itemize}
	\item[(a)] If we suddenly remove the partition, what happens? What is the final temperature and pressure of the gas?
	\item[(b)] (Challenge) If we instead slowly move the partition to the right until it joins the rightmost wall, what happens to the gas along the way? (Qualitatively) What is the final temperature and pressure of the gas?
\end{itemize}

\end{document}