\documentclass{article}

\usepackage{amsmath}

\begin{document}
	
\title{Probability Distributions and Gases}
\author{GSI: Caleb Eades}
\date{8/30}
\maketitle

\section{Ideal Gas Law}

\subsection{Pressured balloons}

The lake adds a pressure $\rho gh$ where $h$ is the height beneath the surface. So letting $D$ bet the depth of the lake,
\begin{align*}
PV &= nRT \\
P' &= nRT'/V' \\
P_{balloon} + P_{depth} &= nRT'/V' \\
nRT/V + \rho gh &= nRT'/V' \\
\rho gh &= nR(T'/V'-T/V) \\
h &= \frac{nR}{\rho g}(T'/V'-T/V) \\
&= \frac{PV}{\rho gT}(T'/V'-T/V) \\
&= \frac{P}{\rho g}\left(\frac{VT'}{TV'}-1\right)
\end{align*}

\subsection{Pressure gauges}

Have fun!

\subsection{Estimations with the ideal gas law}

In a standard room, $P\approx 1 atm$, $T \approx 300 K$, $R\approx 8 J mol^{-1} K^{-1}$, $V\approx (5 m)^2(4 m) = 100 m^3$, and $1 atm\approx 10^5 N m^{-2}$. So putting all this together,
\begin{equation}
n = \frac{PV}{RT} \approx \frac{(10^5 N m^{-2})(100 m^3)}{(8 N m mol^{-1} K^{-1})(300 K)} \approx \frac{100}{25}\times 10^3 mol = 4000 moles
\end{equation}
Multiplying by Avogadro's number $\left(6.022\times10^{23}\right)$, we get
\begin{equation}
N\approx 2.5\times10^{26}
\end{equation}

\subsection{Partitioned boxes}

\begin{itemize}
	\item[(a)] The gas expands into the remainder of the box that used to be a vacuum. So $V/2\rightarrow V$ and by the Ideal Gas Law,
	\begin{align*}
	P_0\frac{V}{2}&=nRT_0 \\
	P_fV &= nRT_f
	\end{align*}
	The termpreature remains unaltered, however, because nothing has been doen to increase or decrease the kinetic energy of the molecules, so $T_f=T_0$ and therefore by the two equations above, $P_f=P_0/2$.
	\item[(b)] This is trickier because along the way, something has to hold the wall in place, which does negative work on the gas (you can also think about the gas doing work on the wall to expand and push it outwards), so the temperature no longer remains constant. Hence, $T$ decreases, $V$ increases ($V/2\rightarrow V$ as before) and $P$ decreases (by more than half now).
\end{itemize}

\newpage

\section{Probability Distributions}

\subsection{Statistics on distributions}

\begin{itemize}
	\item[(a)] We simply integrate the distribution over all possible values of $v$ and enforce that it must equal one (to be a valid probability distribution):
	\begin{equation}
	\int_{-\sqrt{a/b}}^{\sqrt{a/b}}f(v)dv = 1 \implies \left[ Cav - \frac{Cb}{3}v^3\right|_{-\sqrt{a/b}}^{\sqrt{a/b}} = 1
	\end{equation}
	Plugging in the numbers, we have
	\begin{align*}
	Ca(2\sqrt{a/b})-\frac{Cb}{3}\frac{a}{b}(2\sqrt{a/b}) &= 1 \\
	C &= \frac{3}{4a}\sqrt{\frac{b}{a}}
	\end{align*}
	\item[(b)] To compute average velocity, we integrate the distribution weighted by the velocity over the full range, using the value for $C$ that we cound in part (a):
	\begin{align*}
	\bar{v} &= \int_{-\sqrt{a/b}}^{\sqrt{a/b}}vf(v)dv \\
	&= \int_{-\sqrt{a/b}}^{\sqrt{a/b}}(Cav-Cbv^3)dv \\
	&= \left[ \frac{Ca}{2}v^2 - \frac{Cb}{4}v^4\right|_{-\sqrt{a/b}}^{\sqrt{a/b}} \\
	&= \frac{Ca}{2}\left(\frac{a}{b}-\frac{a}{b}\right) - \frac{Cb}{4}\left(\left(\frac{a}{b}\right)^2-\left(\frac{a}{b}\right)^2\right) \\
	&= 0
	\end{align*}
	Equivalently, we could have immediately concluded $\bar{v} = 0$ m/s by simply saying integrating an odd function over a domain symmetric about the origin always yields zero.
	\item[(c)] We proceed exactly as in part (b), except with $v^2$ as our weight on the probability distribution:
	\begin{align*}
	\bar{v^2} &= \int_{-\sqrt{a/b}}^{\sqrt{a/b}}v^2f(v)dv \\
	&= \int_{-\sqrt{a/b}}^{\sqrt{a/b}}(Cav^2-Cbv^4)dv \\
	&= \left[ \frac{Ca}{3}v^3 - \frac{Cb}{5}v^5\right|_{-\sqrt{a/b}}^{\sqrt{a/b}} \\
	&= C\left[\frac{a}{3}\sqrt{\frac{a}{b}}\frac{2a}{b}-\frac{b}{5}\sqrt{\frac{a}{b}}\frac{2a^2}{b^2}\right] \\
	&= \left(\frac{3}{4a}\sqrt{\frac{b}{a}}\right)\left(\frac{2a^2}{b}\sqrt{\frac{a}{b}}\right)\left(\frac{1}{3}-\frac{1}{5}\right) \\
	&= \frac{3a}{2b}\frac{2}{15} \\
	&= \frac{a}{5b}
	\end{align*}
	\item[(d)] Quite simply, the number distribution is the probability distribution of the velocities multiplied by the number of particles:
	\begin{equation}
	N(v) = Nf(v)
	\end{equation}
\end{itemize}

\subsection{Concrete numbers}

\begin{itemize}
	\item[(a)] Proceeding formulaicly:
	\begin{align*}
	\bar{v} &= \frac{2*10 + 4*12 + 2*14 + 1*15 + 1*17}{10} \\
	&= 12.8 m/s \\
	\sqrt{\bar{v^2}} &= \sqrt{\frac{2*10^2 + 4*12^2 + 2*14^2 + 1*15^2 + 1*17^2}{10}} \\
	&= 12.96 m/s \\
	&\approx 13.0 m/s
	\end{align*}
	\item[(b)] The average kinetic energy is related to both temperature and the average squared speed velocity via
	\begin{align*}
	\bar{K} &= \frac{1}{2}m\bar{v^2} = \frac{3}{2}k_B T \\
	T &= \frac{1}{3}\frac{m}{k_B}\bar{v^2} \\
	&\approx \frac{13^2}{3}\frac{m}{k_B}\times\frac{m^2}{s^2}
	\end{align*}
	Note that I explicitly keep the units of $m^2 s^{-2}$ since otherwise they would be lost with the $13^2$ being put into $\bar{v^2}$.
\end{itemize}

\subsection{Setting up expressions}

\begin{itemize}
	\item[(a)] Let $f(v)$ be the usual Maxwell distribution. Then,
	\begin{equation}
	0.1 = \left(\int_{\alpha}^{\infty} f(v)dv\right)/N
	\end{equation}
	\item[(b)] This is given by
	\begin{equation}
	p_E = \frac{\int_{\alpha}^{\infty} v^2f(v)dv}{\int_0^{\infty} v^2f(v)dv}
	\end{equation}
	\item[(c)] $N\rightarrow 0.9N$ and $E\rightarrow 0.72E$, so with $T_0 = \frac{2}{3k_B}\frac{E}{N}$ and $T_1 = \frac{2}{3k_B}\frac{0.72E}{0.9N}$, we have
	\begin{align*}
	T_1 &= \frac{0.72}{0.9}T_0 \\
	&= \frac{8}{10}T_0
	\end{align*}
	\item[(d)] We want $\left(\frac{8}{10}\right)^n T_0 < \frac{1}{2}T_0$, where $n$ is the number of evaporative cooling cycles:
	\begin{align*}
	\left(\frac{8}{10}\right)^n &< \frac{1}{2} \\
	n\log\left(\frac{8}{10}\right) &< \log\left(\frac{1}{2}\right) \\
	-n\log\left(\frac{10}{8}\right) &< -\log(2) \\
	n &> \log(2)/\log\left(\frac{10}{8}\right) \\
	n &> 3.1
	\end{align*}
	Hence, we must go through at least four cycles.
\end{itemize}

\subsection{Puddles!}

Puddles evaporate because the molecules move with a distribution of velocities and the ones with the top velocities could have enough to escape the surface tension of the puddle and go into the atmosphere. In a sealed jar, the water cannot evaporate because those fast molecules that espace will ``bounce'' off the walls and go right back in.

\subsection{Building intuition}

As $N$ increases (keeping $T$ constant), the distribution just gets vertically stretched since it is a number distribution, so this really isn't terribly insightful. As $T$ increases (keeping $N$ constant), the distribution gets a smaller peak but the tail increases, so you get on average faster molecules as the tail with the high speed ones becomes more probable/populated.

\end{document}